
\chapter*{\hfill LỜI NÓI ĐẦU \hfill}
\phantomsection
\addcontentsline{toc}{chapter}{LỜI NÓI ĐẦU}

\paragraph{}
Các hệ thống nhúng đã hiện diện trong hầu hết các hoạt động của con người trong cuộc sống ngày nay, từ các thiết bị điện đơn giản đến các thiết bị phức tạp. Để phát triển các hệ thống nhúng như vậy yêu cầu sự hỗ trợ nhất định phần cứng. Dòng vi điều khiển STM32 sử dụng lõi ARM đã và đang được sử dụng rộng rãi trong việc phát triển các hệ thống nhúng nói trên do hỗ trợ nhiều loại ngoại vi thông dụng như GPIO, TWI, SPI, ADC và có nhiều thư viện hỗ trợ từ ARM Holdings cũng như ST Microelectronics.

\paragraph{}
Với mong muốn tìm hiểu sâu hơn về hệ sinh thái ARM và dòng vi điều khiển STM32 dựa trên lõi ARM Cortex M3, nhóm chúng em quyết định thực hiện bài tập lớn: “\textbf{Lập trình vi xử lý bằng chip STM32F103C8T6 theo yêu cầu: Lập trình giao tiếp UART để Viết ứng dụng điều khiển thực hiện gửi chuỗi ký tự "Ho va ten cua sinh vien" lên PC, sau đó nhận ký tự được gửi từ PC rồi hiển thị ký tự nhận được lên LCD1602}”. Báo cáo bao gồm các chương và bố cục sau:

\hspace{1cm}Chương 1: Cơ sở lý thuyết

\hspace{1cm}Chương 2: Thực hiện đề tài

\hspace{1cm}Chương 3: Kết luận

\vspace{1cm}

\hspace{6cm}Hà Nội, ngày 06 tháng 6 năm 2022

\vspace{2cm}

\hspace{7cm}Nhóm sinh viên thực hiện

