
\chapter*{LỜI NÓI ĐẦU}
\phantomsection
\addcontentsline{toc}{chapter}{LỜI NÓI ĐẦU}


Các hệ thống nhúng đã hiện diện trong hầu hết các hoạt động của con người trong cuộc sống ngày nay, từ các thiết bị điện đơn giản đến các thiết bị phức tạp. Phát triển các hệ thống nhúng như vậy, yêu cầu lượng thời gian nghiên cứu cũng như lượng kiến thức lớn gấp bội khi độ phức tạp của hệ thống tăng lên. Để đơn giản hóa quy trình phát triển cũng như chi phí xây dựng hệ thống, các hệ thống nhúng nói trên có thể phát triển trên nền hệ điều hành linux.


Với mong muốn tìm hiểu sâu hơn về hệ sinh thái mã nguồn mở linux và quy trình phát triển hệ thống nhúng trên nền linux, nhóm chúng em quyết định thực hiện đề tài: “\textbf{Xây dựng ứng dụng với giao diện đồ họa nhập vào dòng ký tự, sau đó hiển thị dòng ký tự đó lên màn hình LCD và giao diện ứng dụng}”. Báo cáo bao gồm các chương và bố cục sau:

\hspace{1cm}Chương 1: Cơ sở lý thuyết

\hspace{1cm}Chương 2: Phân tích - thiết kế

\hspace{1cm}Chương 3: Xây dựng hệ thống

\hspace{1cm}Chương 3: Thử nghiệm và đánh giá

\vspace{0.5cm}

\hspace{6cm}Hà Nội, ngày 06 tháng 6 năm 2022

\vspace{2cm}

\hspace{7cm}Nhóm sinh viên thực hiện

