
Hệ thống nhúng (Embedded system) là một thuật ngữ để chỉ một hệ thống có khả năng tự trị được nhúng vào một môi trường hay một hệ thống mẹ. Đó là các hệ thống tích hợp cả phần cứng và phần mềm phục vụ các bài toán chuyên dụng trong nhiều lĩnh vực công nghiệp, tự động hoá điều khiển, quan trắc và truyền tin. Đặc điểm của các hệ thống nhúng là hoạt động ổn định và có tính năng tự động hoá cao. \cite{giao-trinh}

Hệ thống nhúng thường được thiết kế để thực hiện một chức năng chuyên biệt nào đó. Khác với các máy tính đa chức năng, chẳng hạn như máy tính cá nhân, một hệ thống nhúng chỉ thực hiện một hoặc một vài chức năng nhất định, thường đi kèm với những yêu cầu cụ thể và bao gồm một số thiết bị máy móc và phần cứng chuyên dụng mà ta không tìm thấy trong một máy tính đa năng nói chung. Vì hệ thống chỉ được xây dựng cho một số nhiệm vụ nhất định nên các nhà thiết kế có thể tối ưu hoá nó nhằm giảm thiểu kích thước và chi phí sản xuất. Các hệ thống nhúng thường được sản xuất hàng loạt với số lượng lớn. Hệ thống nhúng rất đa dạng, phong phú về chủng loại. Xét về độ phức tạp, hệ thống nhúng có thể rất đơn giản với một vi điều khiển hoặc rất phức tạp với nhiều đơn vị, các thiết bị ngoại vi và mạng lưới được nằm gọn trong một lớp vỏ máy lớn.

Các thiết bị PDA hoặc máy tính cầm tay cũng có một số đặc điểm tương tự với hệ thống nhúng như các hệ điều hành hoặc vi xử lý điều khiển chúng nhưng các thiết bị này không phải là hệ thống nhúng thật sự bởi chúng là các thiết bị đa năng, cho phép sử dụng nhiều ứng dụng và kết nối đến nhiều thiết bị ngoại vi.

Có rất nhiều hãng sản xuất bộ vi xử lý và phần cứng khác nhau: Texas Instrument, Freescale, ARM, Intel, Motorola, Atmel, AVR, …

Những hệ điều hành khác nhau: QNX, ulTRON, Windows CE/XP Embedded, Embedded Linux, …

Những ngôn ngữ lập trình khác nhau: C/C++, Assembly, PLC, …
