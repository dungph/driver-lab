Nhân Linux (Linux Kernel): Nhân Linux là thành phần cốt lõi của hệ điều hành Linux. Nó quản lý tài nguyên phần cứng, quản lý bộ nhớ, giao tiếp với thiết bị và cung cấp các dịch vụ hệ thống cần thiết. Nhân Linux có thể được tùy chỉnh để phù hợp với các yêu cầu của hệ thống nhúng và được cung cấp bởi dự án Linux Kernel.

Bootloader: Bootloader là chương trình đầu tiên được chạy khi hệ thống được khởi động. Nó có nhiệm vụ khởi động và tải nhân Linux vào bộ nhớ và sau đó chuyển quyền kiểm soát cho nhân Linux. Các bootloader phổ biến cho hệ thống nhúng Linux bao gồm U-Boot và GRUB.

Thư viện hỗ trợ: Hệ thống nhúng Linux cung cấp các thư viện hỗ trợ như libc, libpthread, libm và nhiều thư viện khác. Những thư viện này cung cấp các hàm, giao diện và tài nguyên hỗ trợ cho việc phát triển ứng dụng.

Công cụ phát triển: Hệ thống nhúng Linux cung cấp các công cụ phát triển mạnh mẽ. Đây là thành phần quan trọng giúp biên dịch các chương trình C tới đúng định dạng nhị phân của hệ thống đích bao gồm:
\begin{itemize}
	\item Compiler: GNU C Compiler.
	\item C library: Bao gồm các header file cho phép giao tiếp với hệ điều hành.
	\item Gdb: Hỗ trợ gỡ lỗi.
	\item Hệ thống Makefile
\end{itemize}

Giao diện người dùng: Giao diện người dùng trong hệ thống nhúng Linux có thể được phát triển bằng sử dụng các framework như Qt hay GTK+. Các giao diện người dùng có thể được hiển thị trên màn hình, giao tiếp thông qua các thiết bị nhập liệu và hiển thị kết quả của ứng dụng.

