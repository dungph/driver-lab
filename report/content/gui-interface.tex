Khi lập trình giao diện đồ họa trong hệ thống nhúng linux, việc lựa chọn công nghệ phổ biến sẽ giúp quá trình phát triển trở nên dễ dàng hơn với bộ thư viện hỗ trợ rộng lớn, hỗ trợ từ cộng đồng cũng nhiều hơn.

Qt là một framework phát triển ứng dụng đa nền tảng rất mạnh mẽ và phổ biến. Nó cung cấp các công cụ và thư viện để phát triển giao diện người dùng, bao gồm cả hỗ trợ cho các thiết bị nhúng. Lập trình viên có thể sử dụng Qt để xây dựng giao diện người dùng đẹp và tương tác trong hệ thống nhúng Linux.

Qt được sử dụng để phát triển giao diện người dùng đồ họa (GUI) và các ứng dụng đa nền tảng chạy trên tất cả các nền tảng máy tính để bàn lớn và hầu hết các nền tảng di động hoặc nhúng. Hầu hết các chương trình GUI được tạo bằng Qt đều có giao diện tự nhiên, trong trường hợp này Qt được phân loại là widget toolkit. Ngoài ra các chương trình không phải GUI cũng có thể được phát triển, chẳng hạn như các công cụ dòng lệnh và consoles cho server. Một ví dụ về một chương trình không phải GUI sử dụng Qt là khung công tác web Cutelyst.

Qt hỗ trợ các trình biên dịch khác nhau, bao gồm trình biên dịch GCC C++ và bộ Visual Studio và có hỗ trợ quốc tế hóa rộng rãi. Qt cũng cung cấp Qt Quick, bao gồm một ngôn ngữ kịch bản lệnh được gọi là QML cho phép sử dụng JavaScript hoặc nhiều ngôn ngữ khác để cung cấp logic. Với Qt Quick, việc phát triển ứng dụng nhanh chóng cho các thiết bị di động trở nên khả thi, trong khi logic vẫn có thể được viết bằng mã gốc để đạt được hiệu suất tốt nhất có thể.

Các tính năng khác bao gồm truy cập cơ sở dữ liệu SQL, phân tích cú pháp XML, phân tích cú pháp JSON, quản lý luồng và hỗ trợ mạng. 

Một số lợi thế của bộ công cụ QT:
\begin{itemize}
	\item Phát triển đa nền tảng: Qt cho phép nhà phát triển viết mã một lần và triển khai trên nhiều nền tảng, bao gồm Windows, macOS, Linux, Android và iOS.
	
	\item Qt Widgets và Qt Quick: Qt cung cấp hai phương pháp chính để xây dựng giao diện người dùng. Qt Widgets là một framework trưởng thành và mạnh mẽ để tạo ra các ứng dụng trên máy tính để bàn, trong khi Qt Quick là một framework hiện đại và khai báo để tạo ra giao diện người dùng mượt mà và linh hoạt, phù hợp cho các nền tảng di động và nhúng.
	
	\item Tín hiệu và khe cắm: Cơ chế tín hiệu và khe cắm của Qt cho phép lập trình dựa trên sự kiện, giúp kết nối các phần khác nhau của ứng dụng và phản ứng với hành động người dùng hoặc sự kiện hệ thống.
	
	\item Thư viện phong phú: Qt cung cấp một loạt thư viện và mô-đun, bao gồm hỗ trợ cho giao tiếp mạng, phân tích XML, tích hợp cơ sở dữ liệu (Qt SQL), đa phương tiện (Qt Multimedia), đồ họa 2D/3D (Qt Graphics View) và nhiều hơn nữa.
	
	\item Môi trường phát triển tích hợp (IDE): Qt cung cấp một môi trường phát triển tích hợp được gọi là Qt Creator, cung cấp các tính năng như chỉnh sửa mã, gỡ lỗi, quản lý dự án và một trình thiết kế giao diện người dùng đồ.
\end{itemize}