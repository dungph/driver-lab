Một hệ thống nhúng luôn có 3 thành phần:
\begin{itemize}
	\item Hardware (Phần cứng): Là phần quan trọng nhất của hệ thống nhúng, nó quyết định chính đến chất lượng, giá thành của hệ thống nhúng.
	\item Operation System (Hệ điều hành): Là phần mềm chạy đầu tiên trong hệ thống cho chức năng điều khiển, quản lý tài nguyên của hệ thống và cung cấp một giao diện phát triển chung.
	\item Applications (Phần mềm ứng dụng): Quyết định chức năng của hệ thống và là thành phần tương tác với người dùng. Trong Linux, nó được gọi là phần mềm nhúng Linux
\end{itemize}

Hệ điều hành nhúng Linux (Linux Embedded Software) là những hệ điều hành nhúng sử dụng nhân Linux, có chức năng điều khiển phần cứng hoặc thiết bị kết nối với hệ thống. Không giống với những phần mềm ứng dụng (Application Software) có thể chạy nhiều hệ thống máy tính, phần mềm nhúng Linux chỉ có thể chạy được trên những phần cứng cố định. Phần mềm nhúng Linux phải được thiết kế để chạy được trên những hệ thống hạn chế về tốc độ xử lý, bộ nhớ… mà vẫn đảm bảo hiệu năng.
